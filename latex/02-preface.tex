
\chapter{}

\section{About}

Hi, my name is Sven\cite{hc01} and I am eager to develop for the Jolla smartphone. On my way some questions came up and I tried to answer them as good as I can. After a while I decided to write down what I've learned, so that I had a central place to come back to and maybe others can benefit from this small document, too. The latest and greatest version is to be found on Github\cite{hc02}. For those who don't want to clone the git project there will be just the pdf version at \url{http://hardcodes.de/SailfishOS/Developing-with-SailfishOS.pdf}.
\\
\\
\emph{This work is licensed under the Creative Commons Attribution-NonCommercial-ShareAlike 4.0 International License. To view a copy of this license, visit \url{http://creativecommons.org/licenses/by-nc-sa/4.0/deed.en_US}.}
\\
\\
This goes only for content I have written myself, I don't claim ownership or copyright/left on cited content. Content from other parties remains under its original license.

All mentioned trademarks and trade names are the property of their respective holders, I have not marked them with a \texttrademark, \textregistered \hspace{4 pt} or \copyright \hspace{4 pt} sign. Use some common sense here. Not all information is written by myself, I've tried to quote as responsible as possible and quite intensive if appropiate. Read these notes as a human being, not like a lawyer.

If you find typos, errors or quirks, have suggestions how to make this document better, please drop me a note. Or collaborate.\hspace{4 pt}\verb,#jolla2gether,!

Don't panic if you are not comfortable with writing in \LaTeX, I am happy to use your Libre/Open Office or even Word documents. The source will be converted to \verb,MultiMarkDown, in near future\footnote{whatever \emph{near} means.} to lower the barrier for contribution. This way you can use your editor of choice and must not learn too many obscure instructions. 
%
